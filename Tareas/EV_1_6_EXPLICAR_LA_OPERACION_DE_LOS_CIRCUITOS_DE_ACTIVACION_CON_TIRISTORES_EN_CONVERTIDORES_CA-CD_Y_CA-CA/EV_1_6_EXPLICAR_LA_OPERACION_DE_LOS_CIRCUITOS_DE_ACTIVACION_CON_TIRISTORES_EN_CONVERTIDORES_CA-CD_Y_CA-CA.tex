\documentclass[12pt]{article}

\begin{document}
ACTIVACION DEL TIRISTOR
Un tiristor se activa incrementándola corriente del ánodo. Esto se puede llevar a cabo  mediante una de las siguientes formas.
TERMICA.   Si la temperatura de un tiristor es alta habrá un aumento en el número de pares electrón-hueco, lo que aumentará las corrientes de fuga. Este aumento en las corrientes hará que      1 y     2  aumenten. Debido a la acción regenerativa (   1+   2) puede tender a la unidad y el tiristor pudiera activarse. Este tipo de activación puede causar una fuga térmica que por lo general se evita.
LUZ.   Si se permite que la luz llegue a las uniones de un tiristor, aumentaran los pares electrón-hueco pudiéndose activar el  tiristor. La activación de tiristores por luz se logra permitiendo que esta llegue a los discos de silicio.
ALTO VOLTAJE.  Si el voltaje directo ánodo  a cátodo es mayor que el voltaje de ruptura directo VBO, fluirá una corriente de fuga suficiente para iniciar una activación regenerativa. Este tipo de activación puede resultar destructiva por lo que se debe evitar.
dv/dt.     Si la velocidad de elevación del voltaje ánodo-cátodo es alta, la corriente de carga de las uniones capacitivas puede ser suficiente para activar el tiristor. Un valor alto de corriente de carga puede dañar el tiristor por lo que el dispositivo debe protegerse contra dv/dt alto. Los fabricantes especifican el dv/dt máximo permisible de los tiristores.
CORRIENTE DE COMPUERTA.     Si un tiristor está polarizado en directa, la inyección de una corriente de compuerta al aplicar un voltaje positivo de compuerta entre la compuerta y las terminales del cátodo activará al tiristor. Conforme aumenta la corriente de compuerta, se reduce el voltaje de bloqueo directo.

En corriente continua:
Normalmente el SCR se comporta como un circuito abierto hasta que activa su compuerta (GATE) con una pequeña corriente (se cierra el interruptor S) y así este conduce y se comporta como un diodo en polarización directa. Si no existe corriente en la compuerta el tristor no conduce. Lo que sucede después de ser activado el SCR, se queda conduciendo y se mantiene así. Si se desea que el tristor deje de conducir, el voltaje +V debe ser reducido a 0 Voltios.
Si se disminuye lentamente el voltaje (tensión), el tristor seguirá conduciendo hasta que por el pase una cantidad de corriente menor a la llamada "corriente de mantenimiento o de retención", lo que causará que el SCR deje de conducir aunque la tensión VG (voltaje de la compuerta con respecto a tierra no sea cero.
Como se puede ver el SCR , tiene dos estados:
1- Estado de conducción, en donde la resistencia entre ánodo y cátodo es muy baja 
2- Estado de corte, donde la resistencia es muy elevada 


corriente Alterna:
Se usa principalmente para controlar la potencia que se entrega a una carga. (en el caso de la figura es un bombillo o foco
La fuente de voltaje puede ser de 110V c.a., 120V c.a., 240V c.a. , etc.
El circuito RC produce un corrimiento de la fase entre la tensión de entrada y la tensión en el condensador que es la que suministra la corriente a la compuerta del SCR. Puede verse que el voltaje en el condensador (en azul) está atrasado con respecto al voltaje de alimentación (en rojo) causando que el tiristor conduzca un poco después de que el tiristor tenga la alimentación necesaria para conducir. 
Durante el ciclo negativo el tiristor se abre dejando de conducir. Si se modifica el valor de la resistencia, por ejemplo si utilizamos un potenciómetro, se modifica el desfase que hay entre las dos tensiones antes mencionadas ocasionando que el SCR se active en diferentes momentos antes de que se desactive por le ciclo negativo de la señal. y deje de conducir.
\end{document}