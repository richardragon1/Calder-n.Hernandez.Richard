\documentclass{article}

\title { modulacion de ancho de pulso con amp y transistores }
\author{Calderon Hernandez Richard}
\date{22/octubre/2019}

\begin{document}

\maketitle

\section{¿Qué es PWM?}
La modulación por ancho de pulso o PWM (Pulse Width Modulation) se usa para controlar el ancho de una señal digital con el propósito de controlar a su vez la potencia que se entrega a ciertos dispositivos. Modificando el ancho del pulso activo (que está en On) se controla la cantidad de corriente que fluye hacia el dispositivo.
\section{¿Cómo funciona un PWM?}
Un PWM funciona como un interruptor, que constantemente se activa y desactiva, regulando la cantidad de corriente y por ende de potencia, que se entrega al dispositivo que se desea controlar. Éstos dispositivos pueden ser motores CC o fuentes de luz en CC, entre otros.
Si un motor es alimentado con 12 voltios, recibe todo el tiempo la corriente que este pide y entrega la máxima potencia, si es alimentado con 0 voltios, no recibe corriente y no obtiene potencia.
En un sistema PWM el motor recibe corriente por un tiempo y deja de recibirlo por otro, repitiéndose este proceso continuamente. Si se aumenta el tiempo en que el pulso está en nivel alto (12 V en nuestro ejemplo), se entrega más potencia y si se reduce el tiempo entrega menos potencia.

Si el ciclo de trabajo es de 0% significa que no hay tiempo con pulso en nivel alto y el motor está apagado.
Si el ciclo de trabajo es de 25% significa que el 25% del tiempo el pulso está en nivel alto y el 75% en nivel bajo.
Si el ciclo de trabajo es de 50% el tiempo que el voltaje está en bajo es igual al que está en alto.
Si el ciclo de trabajo es de 100%, significa que el pulso está todo el tiempo en nivel alto. Esto causa que el motor esté trabajando al máximo porque recibe corriente todo el tiempo.
\section{¿Qué ventajas tiene la modulación por ancho de pulso?}
La principal ventaja es la eficiencia energética. El circuito que tenga este método de control, entrega a la carga una cantidad de potencia que es proporcional a la potencia que necesita para realizar su trabajo.
Si se necesita aumentar la velocidad de un motor se incrementa la potencia que se le entrega (ciclo de trabajo mayor)
Si se necesita disminuir la velocidad de un motor se disminuye la potencia que se le entrega. (ciclo de trabajo menor)
\section{¿Que aplicaciones tiene el PWM?}
Controles de velocidad variables para motores CC
Dimmers para sistemas de iluminación con LEDs

\end{document}