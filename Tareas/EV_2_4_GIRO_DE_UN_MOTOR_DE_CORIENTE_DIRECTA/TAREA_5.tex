
\documentclass{article}

\title {Giro de un motor en CD}
\author{Calderon Hernandez Richard}
\date{15/octubre/2019}

\begin{document}

\maketitle

\section{Cómo hacer que un motor eléctrico cambie su sentido de giro}
Existen tres tipos básicos de motores eléctricos: los de corriente continua, los monofásicos de corriente alterna y los trifásicos de corriente alterna. Si cambias la polaridad del voltaje, un simple motor de corriente continua girará en sentido inverso. Si intercambias la conexión del bobinado de arranque que un motor monofásico, hará que gire en sentido inverso. Para invertir el sentido de giro en un motor trifásico, deberás intercambiar de lugar una de las fases de alimentación. En todos los casos, es recomendable consultar el manual de operación del fabricante.
\section{PASO 1:}
Desconecta la alimentación del motor de corriente continua. Conecta el cable positivo de la batería al terminal negativo del motor y luego el cable negativo de la batería al terminal positivo. Instala un interruptor de doble polo y doble tiro para intercambiar la conexión entre la batería y el motor. Mediante este interruptor, el motor podrá funcionar de las siguientes maneras: con un sentido de giro hacia adelante, con sentido de giro hacia atrás o se apagará. 

\section{PASO 2:}
Desconecta la alimentación eléctrica de un motor monofásico y su interruptor. Quita la cubierta del motor para tener acceso a los cables del bobinado de arranque. Intercambia las conexiones de los cables 5 y 8 del bobinado de arranque. Consulta la tabla de cableado del motor. Si los cables 1 y 8 y los cables 4 y 5 están conectados el motor girará en sentido contrario a las manecillas del reloj. Desconecta los cables y vuelve a conectar el cable 1 con el 5 y el 4 con el 8.
\section{PASO 3:}
Ajusta todas las conexiones y reinstala la caja o cubierta. Vuelve a conectar el interruptor y dale energía al motor. Asegúrate de que esté girando en la dirección correcta.
\section{PASO 4:}
Desconecta la alimentación eléctrica de un motor trifásico de corriente alterna y velocidad fija. Asegúrate de que el interruptor de encendido y el interruptor automático están desconectados para evitar una electrocución accidental. Quita la cubierta o tapa de inspección y accede a los cables de alimentación eléctrica. Intercambia cualquiera de estos cables: T1 a T2, o T1 a T3, o T2 a T3. Intercambia sólo uno de esos cables.

\section{PASO 5:}
Ajusta los cables y los terminales. Vuelve a poner la tapa de inspección. Vuelve a conectar la energía al motor. Asegúrate de que gira en la dirección deseada.


\end{document}