\documentclass[12pt]{article}

\begin{document}
CALDERON HERNANDEZ RICHARD
MECATRONICA 4/A 
\\[2cm]
EV2 2 Explicarlos Arregos Y Parametros De Los Amplificadores Clase A
\\[2cm]
Amplificador clase A. Son aquellos amplificador cuyas etapas de potencia consumen corrientes altas y continuas de su fuente de alimentación, independientemente de si existe señal de audio o no. 
\\[1cm]
Un amplificador de potencia funciona en clase A cuando la tensión de polarización y la amplitud máxima de la señal de entrada poseen valores tales que hacen que la corriente de salida circule durante todo el período de la señal de entrada.
\\[1cm]
En los amplificadores de clase A no hay nunca corriente de reja (base) por lo que es indiferente decir que el amplificador es de clase A1 o de clase A. Lo contrario ocurre en los amplificadores de clase C donde siempre va a existir corriente de reja (base), en este caso es indiferente decir que el amplificador es de clase C2 o de clase C (a secas).
\\[1cm]
CARACTERISTICAS
Esta amplificación presenta el inconveniente de generar una fuerte y constante emisión de calor. No obstante, los transistores de salida están siempre a una temperatura fija y sin alteraciones. 
En general, se afirma que esta clase de amplificación es frecuente en circuitos de audio y en los equipos domésticos de gama alta, ya que proporcionan una calidad de sonido potente y de muy buena calidad. 
Los amplificador de clase A a menudo consisten en un transistor de salida conectado al positivo de la fuente de alimentación y un transistor de corriente constante conectado de la salida al negativo de la fuente de alimentación. 
La señal del transistor de salida modula tanto el voltaje como la corriente de salida. Cuando no hay señal de entrada, la corriente de polarización constante fluye directamente del positivo de la fuente de alimentación al negativo, resultando que no hay corriente de salida, se gasta mucha corriente. Algunos amplificador de clase A más sofisticados tienen dos transistores de salida en configuración push-pull. 
VENTAJA
La clase A se refiere a una etapa de salida con una corriente de polarización mayor que la máxima corriente de salida que dan, de tal forma que los transistores de salida siempre están consumiendo corriente. La gran ventaja de la clase A es que es casi lineal, y en consecuencia la distorsión es menor. 
DESVENTAJA
La gran desventaja de la clase A es que es poco eficiente, se requiere un amplificador de clase A muy grande para dar 50 W, y ese amplificador usa mucha corriente y se pone a muy alta temperatura. 
\end{document}